% !TeX spellcheck = en_GB
%%%%%%%%%%%%%%%%%%%%%%%%%%%%%%%%%%%%%%%%%%
%                                        %
%    Engineer thesis LaTeX template      % 
%                                        %
%%%%%%%%%%%%%%%%%%%%%%%%%%%%%%%%%%%%%%%%%%



\documentclass[a4paper,twoside,12pt]{book}
\usepackage[utf8]{inputenc}                                      
\usepackage[T1]{fontenc}  
\usepackage{amsmath,amsfonts,amssymb,amsthm}
\usepackage[polish,british]{babel} 
\usepackage{indentfirst}
\usepackage{lmodern}
\usepackage{graphicx} 
\usepackage{hyperref}
\usepackage{booktabs}
%\usepackage{tikz}
%\usepackage{pgfplots}
\usepackage{mathtools}
\usepackage{geometry}
\usepackage[page]{appendix} 
\usepackage[
backend=biber
]{biblatex}

\addbibresource{refs.bib}

\usepackage{setspace}
\onehalfspacing


\frenchspacing

\usepackage{listings}
\lstset{
	language={},
	basicstyle=\ttfamily,
	keywordstyle=\lst@ifdisplaystyle\color{blue}\fi,
	commentstyle=\color{gray}
}

%%%%%%%%%

 

%%%%%%%%%%%% FANCY HEADERS %%%%%%%%%%%%%%%

\usepackage{fancyhdr}
\pagestyle{fancy}
\fancyhf{}
\fancyhead[LO]{\nouppercase{\it\rightmark}}
\fancyhead[RE]{\nouppercase{\it\leftmark}}
\fancyhead[LE,RO]{\it\thepage}


\fancypagestyle{onlyPageNumbers}{%
   \fancyhf{} 
   \fancyhead[LE,RO]{\it\thepage}
}

\fancypagestyle{PageNumbersChapterTitles}{%
   \fancyhf{} 
   \fancyhead[LO]{\nouppercase{\it\rightmark}}
   \fancyhead[RE]{\nouppercase{\it\leftmark}}
   \fancyhead[LE,RO]{\it\thepage}
}


%%%%%%%%%%%%%%%%%%%%%%%%%%%
% listings 
\usepackage{listings}
\lstset{%
language=C++,%
commentstyle=\textit,%
identifierstyle=\textsf,%
keywordstyle=\sffamily\bfseries, %\texttt, %
%captionpos=b,%
tabsize=3,%
frame=lines,%
numbers=left,%
numberstyle=\tiny,%
numbersep=5pt,%
breaklines=true,%
morekeywords={descriptor_gaussian,descriptor,partition,fcm_possibilistic,dataset,my_exception,exception,std,vector},%
escapeinside={@*}{*@},%
%texcl=true, % wylacza tryb verbatim w komentarzach jednolinijkowych
}
%%%%%%%%%%%%%%%%%%%%%%%%%%%%%%%%%%%%

%%%% TODO LIST GENERATOR %%%%%%%%%

\usepackage{color}
\definecolor{brickred}      {cmyk}{0   , 0.89, 0.94, 0.28}

\makeatletter \newcommand \kslistofremarks{\section*{Remarks} \@starttoc{rks}}
  \newcommand\l@uwagas[2]
    {\par\noindent \textbf{#2:} %\parbox{10cm}
{#1}\par} \makeatother


\newcommand{\remark}[1]{%
{%\marginpar{\textdbend}
{\color{brickred}{[#1]}}}%
\addcontentsline{rks}{uwagas}{\protect{#1}}%
}

%%%%%%%%%%%%%% END OF TODO LIST GENERATOR %%%%%%%%%%% 

% some issues...

\newcounter{PagesWithoutNumbers}

\newcommand{\hcancel}[1]{%
    \tikz[baseline=(tocancel.base)]{
        \node[inner sep=0pt,outer sep=0pt] (tocancel) {#1};
        \draw[red] (tocancel.south west) -- (tocancel.north east);
    }%
}%

\newcommand{\MonthName}{%
  \ifcase\the\month
  \or January% 1
  \or February% 2
  \or March% 3
  \or April% 4
  \or May% 5
  \or June% 6
  \or July% 7
  \or August% 8
  \or September% 9
  \or October% 10
  \or November% 11
  \or December% 12
  \fi}


%%%%%%%%%%%%%%%%%%%%%%%%%%%%%%%%%%%%%%%%%%%%%%
% Helvetica font macros for the title page:
\newcommand{\headerfont}{\fontfamily{phv}\fontsize{18}{18}\bfseries\scshape\selectfont}
\newcommand{\titlefont}{\fontfamily{phv}\fontsize{18}{18}\selectfont}
\newcommand{\otherfont}{\fontfamily{phv}\fontsize{14}{14}\selectfont}

%%%%%%%%%%%%%%%%%%%%%%%%%%%%%%%%%%%%%%%%%%%%%%
%%%%%%%%%%%%%%%%%%%%%%%%%%%%%%%%%%%%%%%%%%%%%%
%%%%%%%%%%%%%%%%%%%%%%%%%%%%%%%%%%%%%%%%%%%%%%
%%%%%%%%%%%%%%%%%%%%%%%%%%%%%%%%%%%%%%%%%%%%%%
%%%%%%%%%%%%%%%%%%%%%%%%%%%%%%%%%%%%%%%%%%%%%%
%%%%%%%%%%%%%%%%%%%%%%%%%%%%%%%%%%%%%%%%%%%%%%
%%%%%%%%%%%%%%%%%%%%%%%%%%%%%%%%%%%%%%%%%%%%%%


\newcommand{\Author}{Łukasz Kwiecień}
\newcommand{\Supervisor}{Tomasz Jastrząb, PhD}
\newcommand{\Title}{Design and implementation of web application used for solving vehicle routing problems with time windows.}
\newcommand{\Polsl}{Silesian University of Technology}
\newcommand{\Faculty}{Faculty of Automatic Control, Electronics and Computer Science}
\newcommand{\Programme}{Programme: Informatics}


\begin{document} 
	
%%%%%%%%%%%%%%%%%%  Title page %%%%%%%%%%%%%%%%%%% 
\pagestyle{empty}
{
	\newgeometry{top=2.5cm,%
	             bottom=2.5cm,%
	             left=3cm,
	             right=2.5cm}
	\sffamily
	\rule{0cm}{0cm}
	
	\begin{center}
	\includegraphics[width=45mm]{logo_eng.jpg}
	\end{center} 
	\vspace{1cm}
	\begin{center}
	\headerfont \Polsl
	\end{center}
	\begin{center}
	\headerfont \Faculty
	\end{center}
	\vfill
	\begin{center}
   \headerfont \Programme
	\end{center}
	\vfill
	\begin{center}
	\titlefont Final Project
	\end{center}
	\vfill
	
	\begin{center}
	\otherfont \Title\par
	\end{center}
	
	\vfill
	
	\vfill
	 
	\noindent\vbox
	{
		\hbox{\otherfont author: \Author}
		\vspace{12pt}
		\hbox{\otherfont supervisor: \Supervisor}
		\vspace{12pt}
	}
	\vfill 
 
   \begin{center}
   \otherfont Gliwice,  \MonthName\ \the\year
   \end{center}	
	\restoregeometry
}
  

\cleardoublepage
 

\rmfamily
\normalfont



%%%%%%%%%%%%%%%%%% Table of contents %%%%%%%%%%%%%%%%%%%%%%
\pagenumbering{Roman}
\pagestyle{onlyPageNumbers}
\tableofcontents

%%%%%%%%%%%%%%%%%%%%%%%%%%%%%%%%%%%%%%%%%%%%%%%%%%%%%
\setcounter{PagesWithoutNumbers}{\value{page}}
\mainmatter
\pagestyle{empty}

\chapter*{Abstract}

Abstract - the abstract text should be copied into the respective field in the APD system. Abstract with keywords should not exceed one page.

\bf{Keywords:} 2-5 keywords, separated by commas

\addcontentsline{toc}{chapter}{Abstract}

\cleardoublepage


\pagestyle{PageNumbersChapterTitles}

%%%%%%%%%%%%%% body of the thesis %%%%%%%%%%%%%%%%%


\chapter{Introduction}

 Transportation is one of the most critical activities in the supply chain. Its importance comes from the fact that the transportation costs can reach up to almost 40\% of the total logistics costs of a manufacturing company. \cite{bib:article:sukiennik} For that reason, the companies need to transport the goods or persons efficiently. That goal can be achieved by either minimizing the distance traveled by the vehicles or reducing the number of routes required to visit all destinations. The Vehicle Routing Problem (VRP) describes the problem of assigning loads to the vehicles and sequencing the customers assigned to each vehicle to obtain optimal routes.
\paragraph{}
VRP is one of the most studied combinatorial optimization problems. Its popularity is due to the fact that VRP can be used in many real-life scenarios in the fields of distribution, collection, and logistics.
In a VRP the fleet of vehicles has to visit a set of customers starting from a given depot and deliver commodities to them, taking into consideration all given constraints. There are various constraints that could be introduced to the problem, for example,the time windows for the customers or the capacity of vehicles.
\paragraph{}
Depending on the problem variant, VRP is a combination of two or more NP-hard problems, which makes VRP also NP-hard problem. The fact that the problem is NP-hard makes the process of finding an exact solution very time-consuming. Therefore it is necessary to use a heuristic approach to get good solutions in an acceptable time.
\paragraph{}
In this thesis a web application solving Capacitated Vehicle Routing Problem with Time Windows (CVRTPW) is considered. User defines the problem by picking waypoints on the map and determining the capacity and time window constraints. Application, if feasible, solves the problem and presents the results in a user-friendly way by visualizing all routes on the map.
The problem is solved using Push Forward Insertion Heuristic and optimized using Local Search with {$\lambda$} interchange method. 
The fleet of vehicles is homogeneus, which means that every vehicle has the same amount of goods that can be transported in it. Every customer has its own time window within which the delivery must be made.
The goal was to create a tool that will simplify the process of defining and solving the CVRTPW for the end user.


\section{Content outline}
 The second chapter "Problem analysis" provides history and explores scientific background material for the Capacitated Vehicle Routing Problem with Time Windows. Moreover provides formal and detailed definition of the problem and the solution methods based on the researched literature. Chapter 3 focuses on the design and implementation process of the project. Shows the functional and nonfuctional requirements as well as diagrams describing the system. The tools and metodologies used for the design and implementation process are also described there. The fourth chapter covers the hardware and software requirements for the application. Provides the step by step installation procedure along with the user manual. The usage examples and screenshots of the working application can also be found there. Chapter 5 discusses the software part of the project in detail. Description of the architecture, code and structure of the project is explained in this chapter. Furthermore specifies in detail the software-side of the entire process of solving the CVRPTW. Chapter 6 shows how the project was tested and verified if the requirements set during the design process were fulfilled. The last chapter 'Conclusions' summarizes the achieved results and describes the encountered difficulties.



\chapter{[Problem analysis]}

\begin{itemize}
\item  problem analysis
\item state of the art, problem statement
\item  literature research (all sources in the thesis have to be referenced \cite{bib:article,bib:book,bib:conference,bib:internet})
\item description of existing solutions (also scientific ones, if the problem is scientifically researched), algorithms,  location of the thesis in the scientific domain
\end{itemize}

\section{Vehicle Routing Problem}
The Vehicle Routing Problem was first stated by Dantzig and Ramser in 1959 as "The Truck Dispatching Problem" \cite{bib:article:TruckDispatching}. The problem concerns a set of customers to whom products need to be delivered in such a way that the delivery cost is as low as possible. Transport is performed by a fleet of vehicles, each of which can carry a certain number of goods. The goal is to serve all customers with as few vehicles as possible and to keep the total distance travelled by all vehicles as short as possible within predefined constraints. The classic version of VRP has the following constraints: 
\begin{enumerate}
	\item Each vehicle's route starts and ends at a depot.
	\item All goods must be delivered.
	\item Customer must receive all goods at one time delivered by one vehicle.
	\item The vehicle may not carry more goods than its capacity allows.	
\end{enumerate}
\paragraph{}
The VRP can be formally defined as directed graph $\textit{G = (V,A)}$, where $\textit{V = $\lbrace$0,\ldots,n$\rbrace$}$ is the set of vertices representing customers and the depot, and \textit{A} is the set of arcs \textit{(i,j)} connecting these vertices. The depot is marked as vertex 0. The number of vehicles is denoted by \textit{m}, and each vehicle has a capacity of \textit{Q}. The cost of travel between vertices \textit{i} and \textit{j} is denoted by \textit{$c_{ij}$}. The cost is the distance, duration or other costs that may occur during the travel between nodes. The demand of customer \textit{i} is represented as \textit{$q_{i}$}.
The customer subset \textit{S $\subseteq$ V$\setminus$ $\lbrace$0$\rbrace$} can be served by a minimum of \textit{r(S)} vehicles. This number can be obtained by solving the Bin Packing Problem (BPP) with bin set \textit{S} with capacities \textit{Q}, but because BPP is an NP-hard problem, it can be approximated by its lower bound: $\lceil$ $\sum$\textsubscript{i$\in$S} $q_{i}$/Q $\rceil$\cite{bib:article:CordeauVehicleRouting}.
\paragraph{}
The Integer Linear Programming formulation of the problem\cite{bib:book:TothAndVigo}: 
\begin{equation}
Minimize \sum_{i,j \in V} c_{ij}x_{ij}
\end{equation}

\textit{subject to:}

\begin{equation}
\sum_{i \in V}x_{ij} = 1 \qquad \forall j \in V \setminus \lbrace 0 \rbrace
\end{equation}

\begin{equation}
\sum_{j \in V}x_{ij} = 1 \qquad \forall i \in V \setminus \lbrace 0 \rbrace
\end{equation}

\begin{equation}
\sum_{j \in V}x_{0j} = m 
\end{equation}

\begin{equation}
\sum_{i \in V}x_{i0} = m 
\end{equation}

\begin{equation}
\sum_{i \not\in S} \sum_{j \in S}x_{ij} \geq \lceil \sum_{i \in S} q_{i} / Q \rceil \qquad \forall S \subseteq V \setminus \lbrace 0 \rbrace , S \neq \emptyset
\end{equation}

\begin{equation}
x_{ij} \in \lbrace 0,1 \rbrace \qquad \forall i,j \in V
\end{equation}

In this formula, $x_{ij}$  is a binary variable whose value indicates whether the arc between vertices i and j is traversed in the solution.  
If the connection between these vertices belongs to the solution, then the variable $x_{ij}$ is equal to 1, otherwise it is equal to 0. \textit{Indegree} (2.2) and \textit{outdegree} (2.3) constraints guarantee that every customer is visited only once. Constraints 2.4 and 2.5 ensure that the solution has \textit{m} routes, one for each vehicle (the number of routes does not exceed the number of vehicles).
Constraints 2.6 are called \textit{capacity-cut constraints} and refer to the minimum number of vehicles needed to serve a set of customers whose total demand is $\sum$\textsubscript{i$\in$S} $q_{i}$. These constraints ensure the connectivity of the solution and that there are no paths whose load would exceed the capacity of the vehicle. The last constraint 2.7 ensures the binarity of the variable $x_{ij}$. If $x_{0j}$ is equal to 1 it means that \textit{j} belongs to this route. Any route from the solution can be reconstructed in this way: the next customer in the route is customer \textit{i}, for whom $x_{ij}$ is equal to 1. The last customer is customer i for whom $x_{i0}$ is equal to 1. \cite{bib:book:TothAndVigo}


\chapter{Requirements and tools}

\begin{itemize}
\item functional and nonfunctional requirements
\item use cases (UML diagrams)
\item description of tools
\item methodology of design and implementation
\end{itemize} 


\chapter{External specification}
\begin{itemize}
\item hardware and software requirements
\item installation procedure
\item activation procedure
\item types of users
\item user manual
\item system administration
\item security issues
\item example of usage
\item working scenarios (with screenshots or output files)
\end{itemize}

\begin{figure}
\centering
\includegraphics[width=2cm]{logo_eng.jpg}
\caption{Figure caption (below the figure).}
\label{fig:2}
\end{figure}


\chapter{Internal specification}

\begin{itemize}
\item concept of the system
\item system architecture
\item description of data structures (and data bases)
\item components, modules, libraries, resume of important classes (if used)
\item resume of important algorithms (if used)
\item details of implementation of selected parts
\item applied design patterns
\item UML diagrams
\end{itemize}


Use special environment for inline code, eg \lstinline|descriptor| or \lstinline|descriptor_gaussian|. 
Longer parts of code put in the figure environment, eg. code in Fig. \ref{fig:pseudokod}. Very long listings–move to an appendix.

\begin{figure}
\centering
\begin{lstlisting}
class descriptor_gaussian : virtual public descriptor
{
   protected:
      /** core of the gaussian fuzzy set */
      double _mean;
      /** fuzzyfication of the gaussian fuzzy set */
      double _stddev;
      
   public:
      /** @param mean core of the set
          @param stddev standard deviation */
      descriptor_gaussian (double mean, double stddev);
      descriptor_gaussian (const descriptor_gaussian & w);
      virtual ~descriptor_gaussian();
      virtual descriptor * clone () const;
      
      /** The method elaborates membership to the gaussian fuzzy set. */
      virtual double getMembership (double x) const;
     
};
\end{lstlisting}
\caption{The \lstinline|descriptor_gaussian| class.}
\label{fig:pseudokod}
\end{figure}


\chapter{Verification and validation}
\begin{itemize}
\item testing paradigm (eg V model)
\item test cases, testing scope (full / partial)
\item detected and fixed bugs
\item results of experiments (optional)
\end{itemize}

 
 

\chapter{Conclusions}
\begin{itemize}
\item achieved results with regard to objectives of the thesis and requirements
\item path of further development (eg functional extension …)
\item encountered difficulties and problems
\end{itemize}

 
\begin{table}
\centering
\caption{A caption of a table is \textbf{above} it.}
\label{id:tab:wyniki}
\begin{tabular}{rrrrrrrr}
\toprule
	         &                                     \multicolumn{7}{c}{method}                                      \\
	         \cmidrule{2-8}
	         &         &         &        \multicolumn{3}{c}{alg. 3}        & \multicolumn{2}{c}{alg. 4, $\gamma = 2$} \\
	         \cmidrule(r){4-6}\cmidrule(r){7-8}
	$\zeta$ &     alg. 1 &   alg. 2 & $\alpha= 1.5$ & $\alpha= 2$ & $\alpha= 3$ &   $\beta = 0.1$  &   $\beta = -0.1$ \\
\midrule
	       0 &  8.3250 & 1.45305 &       7.5791 &    14.8517 &    20.0028 & 1.16396 &                       1.1365 \\
	       5 &  0.6111 & 2.27126 &       6.9952 &    13.8560 &    18.6064 & 1.18659 &                       1.1630 \\
	      10 & 11.6126 & 2.69218 &       6.2520 &    12.5202 &    16.8278 & 1.23180 &                       1.2045 \\
	      15 &  0.5665 & 2.95046 &       5.7753 &    11.4588 &    15.4837 & 1.25131 &                       1.2614 \\
	      20 & 15.8728 & 3.07225 &       5.3071 &    10.3935 &    13.8738 & 1.25307 &                       1.2217 \\
	      25 &  0.9791 & 3.19034 &       5.4575 &     9.9533 &    13.0721 & 1.27104 &                       1.2640 \\
	      30 &  2.0228 & 3.27474 &       5.7461 &     9.7164 &    12.2637 & 1.33404 &                       1.3209 \\
	      35 & 13.4210 & 3.36086 &       6.6735 &    10.0442 &    12.0270 & 1.35385 &                       1.3059 \\
	      40 & 13.2226 & 3.36420 &       7.7248 &    10.4495 &    12.0379 & 1.34919 &                       1.2768 \\
	      45 & 12.8445 & 3.47436 &       8.5539 &    10.8552 &    12.2773 & 1.42303 &                       1.4362 \\
	      50 & 12.9245 & 3.58228 &       9.2702 &    11.2183 &    12.3990 & 1.40922 &                       1.3724 \\
\bottomrule
\end{tabular}
\end{table}  

 

 


%%%%%%%%%%%%%%%%%%%%%%%%%%%%%%%%%%%%%%%%%%
\backmatter
\pagenumbering{Roman}
\stepcounter{PagesWithoutNumbers}
\setcounter{page}{\value{PagesWithoutNumbers}}

\pagestyle{onlyPageNumbers}

%%%%%%%%%%% bibliography %%%%%%%%%%%%
%\bibliographystyle{plain}
%\bibliography{bibliography}

\addcontentsline{toc}{chapter}{Bibliography}
\printbibliography
%%%%%%%%%  appendices %%%%%%%%%%%%%%%%%%% 

\begin{appendices} 


 

\chapter*{Index of abbreviations and symbols}
\addcontentsline{toc}{chapter}{Index of abbreviations and symbols}

\begin{itemize}
\item[DNA] deoxyribonucleic acid
\item[MVC] model--view--controller 
\item[$N$] cardinality of data set
\item[$\mu$] membership function of a fuzzy set
\item[$\mathbb{E}$] set of edges of a graph
\item[$\mathcal{L}$] Laplace transformation
\end{itemize}


\chapter*{Listings}
\addcontentsline{toc}{chapter}{Listings}

(Put long listings in the appendix.)

\begin{lstlisting}
partition fcm_possibilistic::doPartition
                             (const dataset & ds)
{
   try
   {
      if (_nClusters < 1)
         throw std::string ("unknown number of clusters");
      if (_nIterations < 1 and _epsilon < 0)
         throw std::string ("You should set a maximal number of iteration or minimal difference -- epsilon.");
      if (_nIterations > 0 and _epsilon > 0)
         throw std::string ("Both number of iterations and minimal epsilon set -- you should set either number of iterations or minimal epsilon.");
   
      auto mX = ds.getMatrix();
      std::size_t nAttr = ds.getNumberOfAttributes();
      std::size_t nX    = ds.getNumberOfData();
      std::vector<std::vector<double>> mV;
      mU = std::vector<std::vector<double>> (_nClusters);
      for (auto & u : mU)
         u = std::vector<double> (nX);
      randomise(mU);
      normaliseByColumns(mU);
      calculateEtas(_nClusters, nX, ds);
      if (_nIterations > 0)
      {
         for (int iter = 0; iter < _nIterations; iter++)
         {
            mV = calculateClusterCentres(mU, mX);
            mU = modifyPartitionMatrix (mV, mX);
         }
      }
      else if (_epsilon > 0)
      {
         double frob;
         do 
         {
            mV = calculateClusterCentres(mU, mX);
            auto mUnew = modifyPartitionMatrix (mV, mX);
            
            frob = Frobenius_norm_of_difference (mU, mUnew);
            mU = mUnew;
         } while (frob > _epsilon);
      }
      mV = calculateClusterCentres(mU, mX);
      std::vector<std::vector<double>> mS = calculateClusterFuzzification(mU, mV, mX);
      
      partition part;
      for (int c = 0; c < _nClusters; c++)
      {
         cluster cl; 
         for (std::size_t a = 0; a < nAttr; a++)
         {
            descriptor_gaussian d (mV[c][a], mS[c][a]);
            cl.addDescriptor(d);
         }
         part.addCluster(cl);
      }
      return part;
   }
   catch (my_exception & ex)                                  
   {                                                       
      throw my_exception (__FILE__, __FUNCTION__, __LINE__, ex.what()); 
   }                                                          
   catch (std::exception & ex)                                 
   {                                                            
      throw my_exceptionn (__FILE__, __FUNCTION__, __LINE__, ex.what()); 
   }                                                            
   catch (std::string & ex)                                     
   {                                                            
      throw my_exception (__FILE__, __FUNCTION__, __LINE__, ex);        
   }                                                             
   catch (...)                                                   
   {                                                             
      throw my_exception (__FILE__, __FUNCTION__, __LINE__, "unknown expection");       
   }  
}
\end{lstlisting} 

\chapter*{List of additional files in electronic submission }
\addcontentsline{toc}{chapter}{List of additional files in electronic submission (if applicable)}

Additional files uploaded to the system include:
\begin{itemize}
																
\item source code of the application,
\item test data,
\item a video showing how software or hardware developed for thesis is used
\item etc.
\end{itemize}
 
\listoffigures
\addcontentsline{toc}{chapter}{List of figures}
\listoftables
\addcontentsline{toc}{chapter}{List of tables}
	

\end{appendices}


\end{document}


%% Finis coronat opus.
